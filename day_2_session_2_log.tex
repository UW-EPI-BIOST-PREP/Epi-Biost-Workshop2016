\documentclass[12pt]{beamer}
\usetheme{default} 

\setbeamertemplate{navigation symbols}{} %gets rid of navigation symbols
\setbeamertemplate{footline}{} %gets rid of bottom navigation bars
\setbeamertemplate{footline}[page number]{} %use this for page numbers

\setbeamertemplate{footline}{%
  \raisebox{5pt}{\makebox[\paperwidth]{\hfill\makebox[10pt]{\scriptsize\insertframenumber~~}}}}

\setbeamertemplate{itemize items}[circle] %round bullet points
\setlength\parskip{10pt} % white space between paragraphs

\usepackage{wrapfig}
\usepackage{subfig}
\usepackage{setspace}
\usepackage{enumerate}
\usepackage{graphicx}
\usepackage{amsmath}
\usepackage{amsfonts}
\usepackage{amssymb}
\usepackage{amsthm}
\usepackage[UKenglish]{isodate}
\usepackage{tikz}
\usepackage{pgfplots}
\usepackage{natbib}
\def\checkmark{\tikz\fill[scale=0.4](0,.35) -- (.25,0) -- (1,.7) -- (.25,.15) -- cycle;} 

% allow drawing arrows
\usetikzlibrary{arrows}
\tikzstyle{arrow}=[draw, -latex] 

% skips
\setlength{\abovecaptionskip}{15pt plus 3pt minus 2pt}
\setlength{\belowcaptionskip}{5pt plus 3pt minus 2pt}
% bracketing shortcuts
\newcommand{\paren}[1]{\left(#1\right)}
\newcommand{\sqbracket}[1]{\left[#1\right]}
\newcommand{\cbracket}[1]{\left\{#1\right\}}
\newcommand{\abs}[1]{\left\lvert#1\right\rvert}
\newcommand{\norm}[1]{\left\lVert#1\right\rVert}
% set up the argmin operator, argmax
\DeclareMathOperator*{\argmin}{arg\,min}
\DeclareMathOperator*{\argmax}{arg\,max}

\newcommand{\myframe}[1]{\begin{frame} \frametitle{#1}}
% the preamble
\title{Day 2, Session 2: Logs/Exponentiation}
\author{Brian Williamson}
\institute{EPI/BIOST Bootcamp 2016}
\date{26 September 2016}

% Start the document
\begin{document}
% The title page
\begin{frame}
\titlepage
\end{frame}

\section{Exponentiation}
\myframe{Exponentiation}
\begin{itemize}
\item A mathematical operation corresponding to repeated multiplication
\item[]
\item The second in the order of operations! (P{\textbf E}MDAS)
\item[]
\item Composed of two numbers: a base, $b$, and an exponent, $n$
\item[]
\item $b^n = \underbrace{b\times b \times \cdots \times b}_\text{$n$ times}$
\end{itemize}
\end{frame}

\myframe{Positive vs negative exponents}
\begin{itemize}
\item Exponents correspond to multiplication
\item[]
\item Positive exponent: multiplication, e.g. $2^2 = 2\times 2$
\item[]
\item Negative exponent: multiplication of reciprocals, i.e. $2^{-2} = \frac{1}{2}\times \frac{1}{2}$
\end{itemize}
\end{frame}

\myframe{Properties of exponents}
\begin{itemize}
\item For any base $b$ and any $n$ an integer:
\begin{itemize}
\item $b^0 = 1$
\item[]
\item $b^1 = b$
\item[]
\item $b^{n+1} = b^n \times b$
\end{itemize}
\item[]
\item For $b \neq 0$ and any $n$ an integer:
\begin{itemize}
\item $b^n = b^{n+1}/b$
\item[]
\item $b^{-n} = 1/b^n$
\end{itemize}
\end{itemize}
\end{frame}

\myframe{Exponent identities}
\begin{itemize}
\item For all $b,c \neq 0$:
\begin{itemize}
\item $b^{m+n} = b^m \times b^n$
\item[]
\item $b^{m \times n} = (b^m)^n$
\item[]
\item $(b\times c)^n = b^n \times c^n$
\end{itemize}
\end{itemize}
\end{frame}

\myframe{Example: integer exponent properties and identities}
\begin{itemize}
\item Take $b = 2$
\item[]
\item $2^0 = 1$
\item[]
\item $2^1 = 2$
\item[]
\item $2^2 = 2 \times 2$
\item[]
\item $2^3 = 2^2 \times 2 = 8$
\item[]
\item $(2 \times 3)^2 = 2^2 \times 3^2 = 4 \times 9 = 36$ (check: $2 \times 3 = 6, 6^2 = 36$)
\end{itemize}
\end{frame}

\myframe{Rational exponents (roots)}
\begin{itemize}
\item $n$th root of $b$: the number $x$ such that $x^n = b$ 
\item[]
\item Written as $b^{1/n}$ or $\sqrt[n]{b}$
\item[]
\item Some identities (for $b$ positive):
\begin{itemize}
\item $b = (b^n)^{1/n}$
\item[]
\item $b^{m/n} = (b^m)^{1/n} = \sqrt[n]{b^m}$
\end{itemize}
\end{itemize}
\end{frame}

\myframe{Exponential function}
\begin{itemize}
\item An important constant: $e$, approximately $2.718$
\item[]
\item Useful as a base for powers
\item[]
\item Define $\exp(x) = e^x$
\item[]
\item Useful identity: $\exp(x+y) = e^{x+y} =  \exp(x)\times \exp(y)$
\end{itemize}
\end{frame}

\myframe{Exercise: exponents and the exponential function}

\end{frame}

\myframe{Solutions: exponents and the exponential function}

\end{frame}

\section{Logs}
\myframe{Logarithms}
\begin{itemize}
\item Exponents correspond to multiplication
\end{itemize}
\end{frame}
\end{document}
